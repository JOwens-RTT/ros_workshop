% !TeX document-id = {ed9f9aeb-6fc7-4042-9422-9ce343a26700}
\documentclass[12pt]{article}

\usepackage{multirow}
\usepackage{multicol}
\usepackage{framed}
%\usepackage{mdframed}
\usepackage{setspace}
\usepackage{color}
\usepackage{xcolor}
\usepackage{hyperref}
\usepackage[pdftex]{graphicx}
\usepackage{lipsum} 
\usepackage[utf8]{inputenc}
\usepackage[english]{babel}
\usepackage{textcomp}
\usepackage[T1]{fontenc} 

\usepackage{enumitem}
\setlist[description]{leftmargin=\parindent,labelindent=\parindent}

\usepackage{geometry}
 \geometry{
 left=20mm,
 right=10mm,
 top=20mm,
 }

\textwidth=7.0in
\topmargin=-0.6in
\leftmargin=0.5in
\textheight=9.25in
\hoffset=-0.5in
\footskip=0.2in

%\usepackage{pygmentize}
\usepackage{minted}
%\newminted[python]{python}{frame=single}
%\fvset{showspaces}
%\renewcommand\FancyVerbSpace{\textcolor{mygray}{\char32}}
\setminted[text]{
	escapeinside=||, 
	%breaksymbolleft=\carriagereturn,
	frame=single,
	%showspaces=true
	framesep=2mm,
	baselinestretch=1.2,
	bgcolor=mygray
}

% all 4 borders
%\newmdenv{allfour}

\hypersetup{
    %bookmarks=true,         % show bookmarks bar?
    unicode=false,          % non-Latin characters in Acrobat’s bookmarks
    pdftoolbar=true,        % show Acrobat’s toolbar?
    pdfmenubar=true,        % show Acrobat’s menu?
    pdffitwindow=false,     % window fit to page when opened
    pdfstartview={FitH},    % fits the width of the page to the window
    pdftitle={My title},    % title
    pdfauthor={Author},     % author
    pdfsubject={Subject},   % subject of the document
    pdfcreator={Creator},   % creator of the document
    pdfproducer={Producer}, % producer of the document
    pdfkeywords={keyword1, key2, key3}, % list of keywords
    pdfnewwindow=true,      % links in new PDF window
    colorlinks=true,       % false: boxed links; true: colored links
    linkcolor=red,          % color of internal links (change box color with linkbordercolor)
    citecolor=green,        % color of links to bibliography
    filecolor=magenta,      % color of file links
    urlcolor=blue           % color of external links
}

%\mdfsetup{
%  linewidth=.5bp,
%  innerleftmargin=3.5bp,
%  innerrightmargin=3.5bp,
%  innertopmargin=.5bp,
%  innerbottommargin=.5bp,
%}



%\pagestyle{myheadings}
%\markright{{\large ME 4140 Fall 2019---The Robotic Operating System}}

\definecolor{monokaibg}{HTML}{272822}
\definecolor{friendlybg}{HTML}{f0f0f0}
\definecolor{mygreen}{rgb}{0.1333 ,  0.5451,    0.1333}
\definecolor{mypink}{rgb}{0.1333 ,  0.5451,    0.1333}
\definecolor{mygreen}{rgb}{0, .39, 0}
%\definecolor{dred}{#8B0000} 
\definecolor{mypurple}{rgb}{0.6,0.1961,0.8}
\definecolor{mybrown}{rgb}{0.5451,0.2706,0.0745}
\definecolor{mygray}{rgb}{.83, .83, .83}

\newcommand{\R}{\color{red}}
\newcommand{\B}{\color{blue}}
\newcommand{\K}{\color{black}}
%\newcommand{\G}{\color{mygreen}}
\newcommand{\PR}{\color{mypurple}}
\newcommand{\GY}{\color{mygray}}

\newcommand{\VA}{\vspace{2mm}}
\newcommand{\VB}{\vspace{5mm}} 
\newcommand{\VC}{\vspace{30mm}} 

\newcommand{\SQ}{\textquotesingle} 

\begin{document}

%\thispagestyle{plain}

%\thispagestyle{plain}

\begin{center}
   {\bf \Large ROS Workshop - Tutorial 2 - Installing ROS}\vspace{3mm} \\
   {\bf \large ME 4140 - Introduction to Robotics - Fall 2020} \vspace{5mm}\\
\end{center}

\begin{description}

\item[\textbf{\underline{Overview:}}] \hfill \vspace{3mm}\\
After completing {\it Tutorial 1 - Virtualize Ubuntu}, your new operating system is running, and you are ready to install ROS. You can read more about this installation \href{http://wiki.ros.org/melodic/Installation/Ubuntu}{here} on the wiki.

\begin{enumerate}
	
	
	\item  Setup your sources.list to accept software from packages.ros.org.

	\begin{minted}{text}
sudo sh -c 'echo "deb http://packages.ros.org/ros/ubuntu \
|\$|(lsb_release -sc) main" > /etc/apt/sources.list.d/ros-latest.list'
	\end{minted}
	
	\item Set up your keys. 
	
	\begin{minted}[]{text}
sudo apt-key adv --keyserver 'hkp://keyserver.ubuntu.com:80'\
--recv-key C1CF6E31E6BADE8868B172B4F42ED6FBAB17C654	
	\end{minted}
				
	\item Update your Linux system. 
	
	\begin{minted}[]{text}
sudo apt update
	\end{minted}
	
	\item Install ROS Melodic Desktop-Full 
	
	\begin{minted}[]{text}
sudo apt install ros-melodic-desktop-full
	\end{minted}


%	\item Initialize rosdep (2 separate commands) 
%	\begin{minted}[]{text}
%sudo rosdep init
%	\end{minted}
%	\begin{minted}[]{text} 
%rosdep update
%	\end{minted}
%
%\newpage
%\item Evironment Setup (2 separate commands) 
%\begin{minted}{text} 
%echo "source /opt/ros/melodic/setup.bash" >> ~/.bashrc
%\end{minted}
%\begin{minted}{text} 
%source ~/.bashrc 
%\end{minted}
%		
%\item Install Development Tools 
%\begin{minted}{text} 
%sudo apt install python-rosinstall python-rosinstall-generator \ 
%python-wstool build-essential
%\end{minted}
%
%		\item Close the terminal. Open a new terminal and try the following command:\\\\
%\begin{minted}{text}  
%roscore
%\end{minted}
%
%		\item If you see ROS start in the terminal your ROS installation was successful!\\\\

\end{enumerate}

\end{description}
\end{document}
