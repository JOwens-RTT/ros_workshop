\documentclass[12pt]{article}

\usepackage{minted}
%\newminted[python]{python}{frame=single}
%\fvset{showspaces}
%\renewcommand\FancyVerbSpace{\textcolor{mygray}{\char32}}
\setminted[text]{
escapeinside=||, 
%breaksymbolleft=\carriagereturn,
frame=single,
%showspaces=true
framesep=2mm,
baselinestretch=1.2,
bgcolor=mygray
}
\setminted[cpp]{
escapeinside=||, 
%breaksymbolleft=\carriagereturn,
frame=single,
%showspaces=true
framesep=2mm,
baselinestretch=1.2,
bgcolor=mygray
}
\usepackage{xcolor}
\usepackage[pdftex]{graphicx}
\usepackage{multicol}
\usepackage{multirow}
\usepackage{setspace}
\usepackage{color}
\usepackage{listings}
\usepackage[T1]{fontenc}
\usepackage[margin=1in]{geometry}
\usepackage{color,soul}
\usepackage{fancyvrb}
\usepackage{framed}
\usepackage{wasysym}
\usepackage{array}
\usepackage{hyperref}
\usepackage{listings}
\usepackage{textcomp}
\usepackage[T1]{fontenc} 
\usepackage{enumitem}


\hypersetup{
    bookmarks=true,         % show bookmarks bar?
    unicode=false,          % non-Latin characters in Acrobat’s bookmarks
    pdftoolbar=true,        % show Acrobat’s toolbar?
    pdfmenubar=true,        % show Acrobat’s menu?
    pdffitwindow=false,     % window fit to page when opened
    pdfstartview={FitH},    % fits the width of the page to the window
    pdftitle={My title},    % title
    pdfauthor={Author},     % author
    pdfsubject={Subject},   % subject of the document
    pdfcreator={Creator},   % creator of the document
    pdfproducer={Producer}, % producer of the document
    pdfkeywords={keyword1, key2, key3}, % list of keywords
    pdfnewwindow=true,      % links in new PDF window
    colorlinks=true,       % false: boxed links; true: colored links
    linkcolor=red,          % color of internal links (change box color with linkbordercolor)
    citecolor=green,        % color of links to bibliography
    filecolor=magenta,      % color of file links
    urlcolor=blue           % color of external links
}

\definecolor{dkgreen}{rgb}{0,0.6,0}
\definecolor{gray}{rgb}{0.5,0.5,0.5}
\definecolor{mauve}{rgb}{0.58,0,0.82}

\definecolor{mygray}{rgb}{.83, .83, .83}
\definecolor{mypurple}{rgb}{0.6,0.1961,0.8}
\definecolor{mybrown}{rgb}{0.5451,0.2706,0.0745}
\definecolor{mygreen}{rgb}{0, .39, 0}

\newcommand{\R}{\color{red}}
\newcommand{\B}{\color{blue}}
\newcommand{\BR}{\color{mybrown}}
\newcommand{\K}{\color{black}}
\newcommand{\G}{\color{mygreen}}
\newcommand{\PR}{\color{mypurple}}

\setulcolor{red} 
\setstcolor{green} 
\sethlcolor{mygray} 

\lstset{frame=tb,
  language=Java,
  aboveskip=3mm,
  belowskip=3mm,
  showstringspaces=false,
  columns=flexible,
  basicstyle={\small\ttfamily},
  numbers=none,
  numberstyle=\tiny\color{gray},
  keywordstyle=\color{blue},
  commentstyle=\color{dkgreen},
  stringstyle=\color{mauve},
  breaklines=true,
  breakatwhitespace=true,
  tabsize=3
}

% ME4140 - Fall 2016

\textwidth=6.5in
\topmargin=-0.5in
\textheight=9.25in
\hoffset=-0.5in
\footskip=0.2in

\pagestyle{myheadings}
%\markright{{\large ME 4140 Fall 2020---The Robotic Operating System}}

\newcommand{\usrname}{\B<user\_name>\K}
\newcommand{\pkgname}{\G<package\_name>\K}
\newcommand{\wspname}{\R<workspace\_name>\K}
\newcommand{\nodname}{\PR<node\_name>\K}
\newcommand{\tpcname}{/<topic\_name>}
\newcommand{\home}{\textasciitilde/}

\newcommand{\rosdistro}{melodic}

\begin{document}

\thispagestyle{plain}

\begin{center}
   {\bf \Large ROS Workshop - Tutorial 4 - Create a Package}\vspace{3mm} \\
   {\bf \large ME 4140 - Introduction to Robotics - Fall 2020} \vspace{5mm}\\
\end{center}

\begin{description}[labelindent=1cm]
	
	\item[\textbf{\underline{Overview:}}] \hfill \vspace{3mm}\\
	After completing {\it Tutorial 3 - Turtlesim}, You have begun learning ROS and you can are ready to create a custom C++ package. You can read more \href{http://wiki.ros.org/catkin/Tutorials/create_a_workspace}{here} on the wiki.
	
	\item[\textbf{\underline{System Requirements:}}] \hfill \vspace{0mm}

\begin{itemize}
	\item {\bf ROS+OS}: This tutorial is intended for a system with ROS Melodic installed on the Ubuntu 18.04 LTS operating system. Alternate versions of ROS (i.e. - Kinetic, Noetic, etc.) may work but have not been tested. Versions of ROS are tied to versions of Ubuntu.
	\item {\bf ROS:} Your computer must be connected to the internet to proceed. Update the system before you begin.
\end{itemize}

	
	\item[\textbf{\underline{Disclaimer:}}] \hfill \vspace{0mm}
	
	\begin{itemize}

		\item {\R\underline{\bf Backup the System:}} If you are using a virtual machine, it is recommend to make a snaphot of your virtual machine before you start each module. In the event of an untraceable error, you can restore to a previous snapshot. 
		
		\item {\bf Workspace Setup:} In Part I you will setup a Catkin Workspace as your working directory for creating packages. {\G \underline {This only needs to be done once.}}  
	\end{itemize}

\begin{framed}
	\item[\underline{Important Note on Naming:}] \hfill \vspace{0mm} 
	
In this tutorial you will replace several <fields> with names of your choice. These are general guidlines for \href{http://wiki.ros.org/ROS/Patterns/Conventions}{naming in ROS}. \vspace{3mm} 
\begin{itemize}
\item use descriptive names, very long or very short names are hard to read
\item \textbf{do not} include the < > symbols
\item \textbf{do not} use spaces, UPPER CASE letters, or special characters $(@,\$,*, etc.)$
\item the underscore \_ character \textbf{is} allowed \vspace{0mm}\\
\end{itemize}
    \wspname \hspace{2mm} - name of your workspace \vspace{3mm}\\
    \pkgname \hspace{2mm} - name of your package  \vspace{3mm}\\
    \nodname \hspace{2mm} - name of your node  \vspace{3mm}\\
	\usrname \hspace{2mm} - ubuntu user name \vspace{3mm}\\
	
	\end{framed}
	\newpage
	\item[\textbf{\underline{ Part I - Setup the \href{http://wiki.ros.org/catkin/Tutorials/create_a_workspace}{Workspace:} }}] {\G ( Part I only needs to be done once Fall2020.) }   \hfill \vspace{0mm}
	
	Before building a custom ROS package you need to setup a {\it catkin workspace} as the working directory. Catkin is the program that manages the file system behind the scenes and compiles your .cpp code. 
	
	\begin{description}
		\item[Step 1:] Source the installation files needed to create a workspace. This requires ROS to be previously installed.
		\begin{minted}{text}
source /opt/ros/|\rosdistro|/setup.bash
		\end{minted}
		
		\item[Step 2:] Open a new terminal and navigate to the future location of your workspace.
		\begin{minted}{text}
cd ~	OR	cd /|\usrname|/home
		\end{minted}
		%{\fontfamily{qcr}\selectfont  \hspace{5mm} \$ cd $\sim$} \hspace{10mm}This is a shortcut for: {\fontfamily{qcr}\selectfont  \hspace{5mm} \$ cd 
		% home/<username>}\\
		
		\item[Step 3:] Choose a workspace name and create a workspace and source directory with {\it mkdir}. This step determines the location of your new workspace.
		\begin{minted}{text}
mkdir -p |\home\wspname|/src
		\end{minted}
		
		
		\item[Step 4:] Navigate to the top of your workspace directory and build your workspace.
		\begin{minted}{text}
cd |\home\wspname|
		\end{minted}
		
		\begin{minted}{text}
catkin_make
		\end{minted}
		
		
		\item [Step 5:]  Now add your workspace directory to .bashrc and source the script.
		\begin{minted}{text} 
echo "source ~/|\wspname|/devel/setup.bash" >> ~/.bashrc
		\end{minted}
		
		\begin{minted}{text} 
source ~/.bashrc 
		\end{minted}
		
		\begin{multicols}{2}
		Open the {\bf .bashrc} file with the gedit text editor. You can see the lines you have added with {\bf echo} >> at the bottom of the file. Close the file.  
		\begin{minted}{text}  
gedit |\home|.bashrc
		\end{minted}
		\end{multicols}
		
		%\item [Step 6:]Before continuing test that your ROS system is setup correctly.
		%\begin{minted}{text} 
		%echo |\$|ROS_PACKAGE_PATH
		%\end{minted}
		%
		%You should see something like this in the terminal. This is the path where ROS is installed. Do not enter this as a command.
		%\begin{minted}{text} 
		%/home/<user_name>/|\wspname|/src:/opt/ros/|\rosdistro|/share
		%\end{minted}
		
	\end{description}
	
	\newpage
	\item[\textbf{\underline{Part II - Create A \href{http://wiki.ros.org/ROS/Tutorials/WritingPublisherSubscriber(c++)}{{\bf Publisher }}Node:}}] \hfill \vspace{0mm}
	   
	   You can write custom nodes for your ROS system in C++, Python, or Lisp. These documents will support C++.
	         \begin{description}    				
	          \item [Step 1:] \href{http://wiki.ros.org/ROS/Tutorials/CreatingPackage}{Create a new package} in your workspace for your new node to belong to. Make sure to do this in the correct parent directory .
	\begin{minted}{text} 
cd |\home\wspname|/src
	\end{minted}
	
	\begin{minted}{text} 
catkin_create_pkg |\pkgname| std_msgs rospy roscpp
	\end{minted}
	            
	            
	\item [Step 2:] Back out to the workspace directory then compile your package with \href{http://wiki.ros.org/catkin/Tutorials/using_a_workspace#Building_Packages_in_a_catkin_Workspace}{catkin\_make} 
	
	\begin{minted}{text}  
cd |\home\wspname| 	OR 	cd ..
	\end{minted}
	             
	\begin{minted}{text}  
catkin_make
	\end{minted}
	            
	
	If you get here with no errors, your workspace is setup, and you are ready to write some code and test your new package!
	            
	            
	            %\end{enumerate}
	\newpage
	\item [Step 3:] Create a new file for your C++ {\bf publisher node} from the command line. The text editor {\it gedit} will create and open a new file named \nodname in the current directory.
	
	%Make sure it is saved in the source directory of the package your created in previously in {\bf step 1}.
	%Write a {\bf publisher node} in C++. It will start as C++ code and then it will be compiled into an executable. Create a  %the sample code shown below. 
	
	\begin{minted}{text}  
gedit |\home\wspname|/src/|\pkgname|/src/|\nodname|.cpp
	\end{minted}
	  
	 Copy the code below into the source file. . \vspace{1mm}
	
	%\begin{minted}{cpp}
	 
	\begin{lstlisting}
	#include "ros/ros.h"
	#include "geometry_msgs/Twist.h"
	#include <sstream>
	
	int main(int argc, char **argv)
	{
	    ros::init(argc, argv, "replace_with_your_node_name");
	    ros::NodeHandle n;
	    ros::Publisher ttu_publisher = n.advertise<geometry_msgs::Twist>("/turtle1/cmd_vel", 1000);
	    ros::Rate loop_rate(10);
	
	    int count = 0;
	    while (ros::ok())
	    {
	        geometry_msgs::Twist msg;
	        msg.linear.x = 2+0.01*count;
	        msg.angular.z = 2;
	        ttu_publisher.publish(msg);
	        ros::spinOnce();
	        loop_rate.sleep();
	        count++;
	    }
	}
	\end{lstlisting}
	
	%\end{minted}
	
	Save and close the file. It must be saved as a \nodname.cpp in the src directory of the package your created in previously in {\bf Part I}
	
	\item[Step 4:] Before we can compile the node we have to modify the file below.
	
	\begin{minted}{text}  
gedit |\home\wspname|/src/|\pkgname|/CMakeLists.txt
	\end{minted}
	
	Add the following lines to the bottom of the file and save.
	
	\begin{minted}[bgcolor=white]{text}
add_executable(|\nodname| src/|\nodname|.cpp)
target_link_libraries(|\nodname| |\$|{catkin_LIBRARIES}) 
	\end{minted}
	\newpage
	 
	
	\item[Step 5:] Compile and test the new publisher node. This will compile and build your source code as well as check for errors in your entire workspace.
	\begin{minted}{text}  
cd |\home\wspname|
	\end{minted}
	
	\begin{minted}{text}  
catkin_make
	\end{minted}
	
	Start a core
	\begin{minted}{text} 
roscore
	\end{minted}
	
	Turn on a turtle.
	\begin{minted}{text} 
rosrun turtlesim turtlesim_node
	\end{minted}
	
	Start your new node
	\begin{minted}{text} 
rosrun |\pkgname\hspace{3mm}\nodname|
	\end{minted}
	
	Use rostopic to view current topics. 
	\begin{minted}{text}  
rostopic list
	\end{minted}
	
	
	Close your node and start it again with the cmd\textunderscore vel topic mapped to the turtle like we did previously.
	\begin{minted}{text} 
rosrun |\pkgname\hspace{3mm}\nodname \hspace{1mm}|/cmd_vel:=/turtle1/cmd_vel
	\end{minted}
	 
	\end{description}
	 
	 
	\newpage
	
	
\item[\textbf{\underline{Part III - Create A \href{http://wiki.ros.org/ROS/Tutorials/WritingPublisherSubscriber(c++)}{{\bf Subscriber }}Node:}}] \hfill \vspace{0mm}
	 	
	Now create a {\bf subscriber node} in the same package as the previous node. 
	
	\begin{description}
	
	\item [Step 1:] Use the code below called {\bf turtlesim\_subscriber.cpp} to start.\\
	
	\begin{lstlisting}
	
	#include "ros/ros.h"
	#include "std_msgs/String.h"
	#include "geometry_msgs/Twist.h"
	/**
	* This tutorial demonstrates simple receipt of messages over the ROS system.
	*/
	void dataCallback(const geometry_msgs::Twist::ConstPtr& msg)
	{
		ROS_INFO("I heard: [%f]", msg->linear.x);
	}
	int main(int argc, char **argv)
	{
		ros::init(argc, argv, "turtlesim_subscriber");
		ros::NodeHandle n;
		ros::Subscriber sub = n.subscribe("/cmd_vel", 1000, dataCallback);
		ros::spin();
		return 0;
	}
	
	\end{lstlisting}
	
	
	\item [Step 2:] Modify the appropriate CMakeLists.txt file as you did previously. \\\\
	%\begin{minted}{text}
	%gedit |\home\wspname/src/\pkgname/CMakeLists.txt|
	%\end{minted}
	
	\item [Step 3:] Compile the new subscriber node using catkin. \\\\
	
	%\begin{minted}{text} 
	%|cd \home\wspname|
	%\end{minted}
	
	%\begin{minted}{text} 
	%catkin_make
	%\end{minted}
	
	\item [Step 4:] Test the new node. Does it work? How do you know?\\
	
	%\begin{minted}{text} 
	%rosrun |\pkgname\hspace{3mm}\nodname| 
	%\end{minted}
	

	\vspace*{5mm}
	


	\end{description}	
	
	
	

	\item[\textbf{\underline{Tutorial Complete:}}] \hfill \vspace{3mm}\\ After completing {\it Tutorial 4 - Create Package}, you are finally ready for a more advanced robot simulator.
	
	\hfill \vspace{3mm}\\
	
	\item [Bonus Excercise:] Install the \href{http://wiki.ros.org/joy/Tutorials/WritingTeleopNode}{JoyStick Teleop Node} to drive the turtle with a USB joystick.
	
	
	\end{description}

\end{document}

