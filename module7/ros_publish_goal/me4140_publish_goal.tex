\documentclass[12pt]{article}

	\addtolength{\oddsidemargin}{-.5in}
	\addtolength{\evensidemargin}{-.875in}
	\addtolength{\textwidth}{1.75in}

	\addtolength{\topmargin}{-.875in}
	\addtolength{\textheight}{1.75in}

\usepackage{minted}
%\newminted[python]{python}{frame=single}
%\fvset{showspaces}
%\renewcommand\FancyVerbSpace{\textcolor{mygray}{\char32}}
\setminted[text]{
escapeinside=||, 
%breaksymbolleft=\carriagereturn,
frame=single,
%showspaces=true
framesep=2mm,
baselinestretch=1.2,
bgcolor=mygray
}
\setminted[cpp]{
escapeinside=||, 
%breaksymbolleft=\carriagereturn,
frame=single,
%showspaces=true
framesep=2mm,
baselinestretch=1.2,
bgcolor=mygray
}
\usepackage{xcolor}
\usepackage{hyperref}
\usepackage[pdftex]{graphicx}
\usepackage{multirow}
\usepackage{setspace}
\usepackage{color}
\usepackage{multicol}
\usepackage{listings}
\usepackage{color}
\usepackage[T1]{fontenc}

\hypersetup{
    bookmarks=true,         % show bookmarks bar?
    unicode=false,          % non-Latin characters in Acrobat’s bookmarks
    pdftoolbar=true,        % show Acrobat’s toolbar?
    pdfmenubar=true,        % show Acrobat’s menu?
    pdffitwindow=false,     % window fit to page when opened
    pdfstartview={FitH},    % fits the width of the page to the window
    pdftitle={My title},    % title
    pdfauthor={Author},     % author
    pdfsubject={Subject},   % subject of the document
    pdfcreator={Creator},   % creator of the document
    pdfproducer={Producer}, % producer of the document
    pdfkeywords={keyword1, key2, key3}, % list of keywords
    pdfnewwindow=true,      % links in new PDF window
    colorlinks=true,       % false: boxed links; true: colored links
    linkcolor=red,          % color of internal links (change box color with linkbordercolor)
    citecolor=green,        % color of links to bibliography
    filecolor=magenta,      % color of file links
    urlcolor=blue           % color of external links
}

\definecolor{dkgreen}{rgb}{0,0.6,0}
\definecolor{gray}{rgb}{0.5,0.5,0.5}
\definecolor{mauve}{rgb}{0.58,0,0.82}
\lstset{frame=tb,
  language=Java,
  aboveskip=3mm,
  belowskip=3mm,
  showstringspaces=false,
  columns=flexible,
  basicstyle={\small\ttfamily},
  numbers=none,
  numberstyle=\tiny\color{gray},
  keywordstyle=\color{blue},
  commentstyle=\color{dkgreen},
  stringstyle=\color{mauve},
  breaklines=true,
  breakatwhitespace=true,
  tabsize=3
 % escapeinside={\%*}{*)}
}

% ME4140 - Fall 2016 - Fall 2017 - Fall 2019



\textwidth=6.5in
\topmargin=-0.5in
\textheight=9.25in
\hoffset=-0.5in
\footskip=0.2in

\pagestyle{myheadings}
\markright{{\large ME 4140 Fall 2019---The Robotic Operating System}}

\definecolor{dkgreen}{rgb}{0,0.6,0}
\definecolor{gray}{rgb}{0.5,0.5,0.5}
\definecolor{mauve}{rgb}{0.58,0,0.82}

\definecolor{mygray}{rgb}{.6, .6, .6}
\definecolor{mypurple}{rgb}{0.6,0.1961,0.8}
\definecolor{mybrown}{rgb}{0.5451,0.2706,0.0745}
\definecolor{mygreen}{rgb}{0, .39, 0}

\newcommand{\R}{\color{red}}
\newcommand{\B}{\color{blue}}
\newcommand{\BR}{\color{mybrown}}
\newcommand{\K}{\color{black}}
\newcommand{\G}{\color{mygreen}}
\newcommand{\PR}{\color{mypurple}}

\newcommand{\pkgname}{<package\_name>}
\newcommand{\wspname}{<workspace\_name>}
\newcommand{\nodname}{<node\_name>}
\newcommand{\tpcname}{<topic\_name>}
\newcommand{\lfname}{<file\_name>}

\newcommand{\home}{\textasciitilde/}

\newcommand{\rosdistro}{melodic}

\newcommand{\pthname}{/opt/ros/\rosdistro/share/turtlebot\_stage/maps/}



\begin{document}

\thispagestyle{plain}

\begin{center}
   {\bf \Large ROS - Publishing and Subscribing with Turtlebot3 }\vspace{2mm} \\
   {\bf \large ME 4140 - Introduction to Robotics - Fall 2019} \\
\end{center}


\begin{enumerate}
    
    \item Create a new node with the name of your choosing. You can put this node in the package you created previously for turtlebot3.  Change directory to the source folder of the package and enter the following command to open a new file for your source code.
\begin{minted}{text}
gedit |\home\wspname|/src/turtlebot3|\_|control/src/publish|\_|goal.cpp
\end{minted}

    

    \item Now copy the example code below into the the source file. Save the file after editing. \\
    
        \begin{lstlisting}
#include "ros/ros.h"
#include "geometry_msgs/PoseStamped.h"
#include <sstream>
int main(int argc, char **argv)
{
    ros::init(argc, argv, "<node_name>");
    ros::NodeHandle n;
    ros::Publisher ttu_publisher =
    n.advertise<geometry_msgs::PoseStamped>("<topic_name>", 1000);
    ros::Rate loop_rate(10);

    geometry_msgs::PoseStamped msg;
    msg.header.stamp=ros::Time::now();
    msg.header.frame_id="map";

    int count = 0;
    while ((ros::ok())&&(count<5))  
    {               
        msg.pose.position.x = 3.0;
        msg.pose.position.y = 2.0;
        msg.pose.position.z = 0;
        msg.pose.orientation.w = 1.0;

        ttu_publisher.publish(msg);
        ros::spinOnce();
        loop_rate.sleep();
        count++;
    }
}
\end{lstlisting}
    
    \newpage
    
    \item Edit the {\it CMakeLists.txt} file for this specific package. \\
   % {\fontfamily{qcr}\selectfont  \hspace{5mm} \$ gedit \home\wspname/src/publish\textunderscore goal/CMakeLists.txt }\\
\begin{minted}{text}
gedit |\home\wspname|/src/publish|\_|goal/CMakeLists.txt
\end{minted}

\vspace{5mm}Add the following lines to the bottom of the file. \\
\underline{\hspace{155mm}}\\\\
{\fontfamily{qcr}\selectfont add\_executable(\nodname\hspace{3mm}src/\nodname.cpp) } \\
{\fontfamily{qcr}\selectfont target\_link\_libraries(\nodname \hspace{3mm}\$\{catkin\_LIBRARIES\}) } \vspace{2mm}\\
\underline{\hspace{155mm}}\\\\

%{\fontfamily{qcr}\selectfont add\_dependencies(\nodname \hspace{3mm}beginner\_tutorials\_generate\_messages\_cpp) }\\

     \item Compile the code with catkin\_make before running. Turn Everything off and navigate to the workspace before you do this.\\
\begin{minted}{text}
cd |\home\wspname|
catkin|\_|make
\end{minted}

    \item Now test your node. Start the turtlebot3 simulator first. 
 
\begin{minted}{text}
roslaunch turtlebot3_gazebo turtlebot3_world.launch
\end{minted}

 \item Next, turn on naviagtion. This requires that you have previously made a map.
\begin{minted}{text}
roslaunch turtlebot3_navigation turtlebot3_navigation.launch map_file:=$HOME/map.yaml
\end{minted}

\item Finally run your new node and you should see your robot move to location programmed in your source file!

\begin{minted}{text}
rosrun turtlebot3|\_|control publish|\_|goal
\end{minted}

     \newpage
\item Now let us make a node called {\bf subscribe\_status} that can access information from the robot in the form of a topic. To do this we are going to make a new node that is part of the same package we just made/used. Open a new file in the proper src folder and insert the following cpp code.
\begin{minted}{text}
gedit |\home\wspname|/src/turtlebot3|\_|control/src/subscribe|\_|status.cpp 
\end{minted}   

         \begin{lstlisting}
#include "ros/ros.h"
#include "std_msgs/String.h"
#include "actionlib_msgs/GoalStatusArray.h"

void statusCB(const actionlib_msgs::GoalStatusArray::ConstPtr& msg)
{
    ROS_INFO("Subscriber Callback Executed");
    if (!msg->status_list.empty())
    {
        actionlib_msgs::GoalStatus goalStatus = msg->status_list[0];
        ROS_INFO("Status Recieved: %i",goalStatus.status);  
    }
}

int main(int argc, char **argv)
{
    ros::init(argc, argv, "subscribe_status");
    ros::NodeHandle n;
    ros::Subscriber sub = n.subscribe("<topic_name>", 1000, statusCB);
    ros::spin();
    return 0;
}
    \end{lstlisting}
    
    \item Open the correct {\it CMakeLists.txt} file for this specific node as you did previously.

\begin{minted}{text}
 gedit |\home\wspname|/src/publish|\_|goal/CMakeLists.txt 
\end{minted}

Add the following lines to the bottom.\\
 \underline{\hspace{155mm}}\\\\
 {\fontfamily{qcr}\selectfont add\_executable(\nodname\hspace{3mm}src/\nodname.cpp) } \\
{\fontfamily{qcr}\selectfont target\_link\_libraries(\nodname \hspace{3mm}\$\{catkin\_LIBRARIES\}) } \\
\underline{\hspace{155mm}}\\\\
 
    \item Compile before running then test you new node! You should see the status information printed in the terminal window.
    
\end{enumerate}
\end{document}

