\documentclass[12pt]{article}

\usepackage{minted}
%\newminted[python]{python}{frame=single}
%\fvset{showspaces}
%\renewcommand\FancyVerbSpace{\textcolor{mygray}{\char32}}
\setminted[text]{
escapeinside=||, 
%breaksymbolleft=\carriagereturn,
frame=single,
%showspaces=true
framesep=2mm,
baselinestretch=1.2,
bgcolor=mygray
}
\setminted[cpp]{
escapeinside=||, 
%breaksymbolleft=\carriagereturn,
frame=single,
%showspaces=true
framesep=2mm,
baselinestretch=1.2,
bgcolor=mygray
}
\usepackage{xcolor}
\usepackage[pdftex]{graphicx}
\usepackage{multicol}
\usepackage{multirow}
\usepackage{setspace}
\usepackage{color}
\usepackage{listings}
\usepackage[T1]{fontenc}
\usepackage[margin=1in]{geometry}
\usepackage{color,soul}
\usepackage{fancyvrb}
\usepackage{framed}
\usepackage{wasysym}
\usepackage{array}
\usepackage{hyperref}
\usepackage{listings}
\usepackage{textcomp}
\usepackage[T1]{fontenc} 

\hypersetup{
    bookmarks=true,         % show bookmarks bar?
    unicode=false,          % non-Latin characters in Acrobat’s bookmarks
    pdftoolbar=true,        % show Acrobat’s toolbar?
    pdfmenubar=true,        % show Acrobat’s menu?
    pdffitwindow=false,     % window fit to page when opened
    pdfstartview={FitH},    % fits the width of the page to the window
    pdftitle={My title},    % title
    pdfauthor={Author},     % author
    pdfsubject={Subject},   % subject of the document
    pdfcreator={Creator},   % creator of the document
    pdfproducer={Producer}, % producer of the document
    pdfkeywords={keyword1, key2, key3}, % list of keywords
    pdfnewwindow=true,      % links in new PDF window
    colorlinks=true,       % false: boxed links; true: colored links
    linkcolor=red,          % color of internal links (change box color with linkbordercolor)
    citecolor=green,        % color of links to bibliography
    filecolor=magenta,      % color of file links
    urlcolor=blue           % color of external links
}

\definecolor{dkgreen}{rgb}{0,0.6,0}
\definecolor{gray}{rgb}{0.5,0.5,0.5}
\definecolor{mauve}{rgb}{0.58,0,0.82}

\definecolor{mygray}{rgb}{.83, .83, .83}
\definecolor{mypurple}{rgb}{0.6,0.1961,0.8}
\definecolor{mybrown}{rgb}{0.5451,0.2706,0.0745}
\definecolor{mygreen}{rgb}{0, .39, 0}

\newcommand{\R}{\color{red}}
\newcommand{\B}{\color{blue}}
\newcommand{\BR}{\color{mybrown}}
\newcommand{\K}{\color{black}}
\newcommand{\G}{\color{mygreen}}
\newcommand{\PR}{\color{mypurple}}

\setulcolor{red} 
\setstcolor{green} 
\sethlcolor{mygray} 

\lstset{frame=tb,
  language=Java,
  aboveskip=3mm,
  belowskip=3mm,
  showstringspaces=false,
  columns=flexible,
  basicstyle={\small\ttfamily},
  numbers=none,
  numberstyle=\tiny\color{gray},
  keywordstyle=\color{blue},
  commentstyle=\color{dkgreen},
  stringstyle=\color{mauve},
  breaklines=true,
  breakatwhitespace=true,
  tabsize=3
}

% ME4140 - Fall 2016

\textwidth=6.5in
\topmargin=-0.5in
\textheight=9.25in
\hoffset=-0.5in
\footskip=0.2in

\pagestyle{myheadings}
\markright{{\large ME 4140 Fall 2019---The Robotic Operating System}}

\newcommand{\usrname}{\B<user\_name>\K}
\newcommand{\pkgname}{\G<package\_name>\K}
\newcommand{\wspname}{\R<workspace\_name>\K}
\newcommand{\nodname}{\PR<node\_name>\K}
\newcommand{\tpcname}{/<topic\_name>}
\newcommand{\home}{\textasciitilde/}

\newcommand{\rosdistro}{melodic}

\begin{document}

\thispagestyle{plain}

\begin{center}
   {\bf \Large ROS - Create a .cpp Node}\vspace{2mm} \\
   {\bf \large ME 4140 - Introduction to Robotics - Fall 2019} \vspace{5mm}\\
\end{center}

\begin{itemize}
    \item  \href{http://wiki.ros.org/Topics}{\bf Topics:} ROS nodes can communicate by {\bf publishing} and {\bf subscribing} to topics. A topic is information generated by a publishing node that is made available to a subscribing node or nodes in the ROS system.
        
        \begin{itemize}		
            \item A node can publish a topic. This node is a publisher.    	
            \item A node can subscribe to a topic. This node is a subscriber.
            \item Most nodes publish and subscribe to multiple topics. 
            \item Integrate built-in ROS nodes and modify your own ROS nodes in C++, Python, and even MATLAB
            
        \end{itemize}

	\item \href{http://wiki.ros.org/catkin/Tutorials/create_a_workspace}{\bf Setup the Workspace:} Before building a custom ROS node you need to setup a {\it catkin workspace}. Catkin is the program that manages the file system behind the scenes and compiles your .cpp code. It is our working directory or environment in which we can customize our ROS system.

	\begin{description}

		\item[Note:] Throughout this tutorial you will need to replace several fields. Do not include the < > symblols.
	
         \wspname \hspace{2mm} - name of your workspace \\
             \pkgname \hspace{2mm} - name of your package \\
            \nodname \hspace{2mm} - name of your node \\
		\usrname \hspace{2mm} - ubuntu user name\\
            



		\item[Step 1:] Source the installation files needed to create a workspace. This requires ROS to be previously installed.
\begin{minted}{text}
source /opt/ros/|\rosdistro|/setup.bash
\end{minted}

\item[Step 2:] Open a new terminal and navigate to the future location of your workspace.
\begin{minted}{text}
cd ~	OR	cd /|\usrname|/home
\end{minted}
%{\fontfamily{qcr}\selectfont  \hspace{5mm} \$ cd $\sim$} \hspace{10mm}This is a shortcut for: {\fontfamily{qcr}\selectfont  \hspace{5mm} \$ cd 
% home/<username>}\\

\item[Step 3:] Choose a workspace name and create a workspace and source directory with {\it mkdir}. This step determines the location of your new workspace.
\begin{minted}{text}
mkdir -p |\home\wspname|/src
\end{minted}

\newpage

\item[Step 4:] Navigate to the top of your workspace directory and build your workspace.
\begin{minted}{text}
cd |\home\wspname|
\end{minted}

\begin{minted}{text}
catkin_make
\end{minted}

\item [Step 5:]Before continuing test that your ROS system is setup correctly.
\begin{minted}{text}
source devel/setup.bash
\end{minted}

\begin{minted}{text} 
echo $ROS_PACKAGE_PATH
\end{minted}

You should see something like this in the terminal. This is the path where ROS is installed. Do not enter this as a command.
\begin{minted}{text} 
/home/<user_name>/|\wspname|/src:/opt/ros/|\rosdistro|/share
\end{minted}

	\end{description}

\newpage
    \item \href{http://wiki.ros.org/ROS/Tutorials/WritingPublisherSubscriber(c++)}{{\bf Create Your Own Node:}} You can write custom nodes for your ROS system in C++, Python, or Lisp. These documents will support C++.
         \begin{description}    				
          \item [Step 1:] \href{http://wiki.ros.org/ROS/Tutorials/CreatingPackage}{Create a new package} in your workspace for your new node to belong to. Make sure to do this in the correct directory .
\begin{minted}{text} 
cd |\home\wspname|/src
\end{minted}

\begin{minted}{text} 
catkin_create_pkg |\pkgname| std_msgs rospy roscpp
\end{minted}
            
            
            
\item [Step 2:] Back out to the workspace directory then compile your package with \href{http://wiki.ros.org/catkin/Tutorials/using_a_workspace#Building_Packages_in_a_catkin_Workspace}{catkin\_make} 

\begin{minted}{text}  
cd |\home\wspname| 	OR 	cd ..
\end{minted}
             
\begin{minted}{text}  
catkin_make
\end{minted}
            
 \item [Step 3:]  Now source the workspace directory.
\begin{minted}{text}   
source |\home\wspname|/devel/setup.bash
\end{minted}

\item[Step 4:]  Open the {\bf .bashrc} file text editor. Modify the file so that this happens 
each time you start a new terminal. 
			
\begin{minted}{text}  
gedit |\home|.bashrc
\end{minted}

\item [Step 5:] Add the following line to the bottom of the file. It may already be there. Save and close the file.		
						
\begin{minted}{text} 
source |\home\wspname|/devel/setup.bash
\end{minted}
%If you get here with no errors you are ready to write some code and test your new package!
            
            
            %\end{enumerate}
     
\newpage
\item [Step 6:] Create a new file for your C++ {\bf publisher node} from the command line. The text editor {\it gedit} will create and open a new file named \nodname in the current directory.

%Make sure it is saved in the source directory of the package your created in previously in {\bf step 1}.
%Write a {\bf publisher node} in C++. It will start as C++ code and then it will be compiled into an executable. Create a  %the sample code shown below. 

\begin{minted}{text}  
gedit |\home\wspname|/src/|\pkgname|/src/|\nodname|.cpp
\end{minted}
  
 Copy the code below into the source file. It must be saved as a \nodname.cpp in the source directory of the package your created in previously in {\bf step 1}. \vspace{1mm}

%\begin{minted}{cpp}
 
\begin{lstlisting}
#include "ros/ros.h"
#include "geometry_msgs/Twist.h"
#include <sstream>

int main(int argc, char **argv)
{
    ros::init(argc, argv, "replace_with_your_node_name");
    ros::NodeHandle n;
    ros::Publisher ttu_publisher = n.advertise<geometry_msgs::Twist>("/turtle1/cmd_vel", 1000);
    ros::Rate loop_rate(10);

    int count = 0;
    while (ros::ok())
    {
        geometry_msgs::Twist msg;
        msg.linear.x = 2+0.01*count;
        msg.angular.z = 2;
        ttu_publisher.publish(msg);
        ros::spinOnce();
        loop_rate.sleep();
        count++;
    }
}
\end{lstlisting}

%\end{minted}

\item[Step 7:] Before we can compile the node we have to modify the file below.

\begin{minted}{text}  
gedit |\home\wspname|/src/|\pkgname|/CMakeLists.txt
\end{minted}

Add the following lines to the bottom of the file and save.

\begin{minted}{text}  
add_executable(|\nodname| src/|\nodname|.cpp)
target_link_libraries(|\nodname| ${catkin_LIBRARIES}) 
add_dependencies(|\nodname| beginner_tutorials_generate_messages_cpp)
\end{minted}
\newpage
 

\item[Step 5:] Compile and test the new publisher node. This will compile and build your source code as well as check for errors in your entire workspace.
\begin{minted}{text}  
cd |\home\wspname|
\end{minted}

\begin{minted}{text}  
catkin_make
\end{minted}

Start a core
\begin{minted}{text} 
roscore
\end{minted}

Turn on a turtle.
\begin{minted}{text} 
rosrun turtlesim turtlesim_node
\end{minted}

Start your new node
\begin{minted}{text} 
rosrun |\pkgname\hspace{3mm}\nodname|
\end{minted}

Use rostopic to view current topics. 
\begin{minted}{text}  
rostopic list
\end{minted}


Close your node and start it again with the cmd\textunderscore vel topic patched through to the turtle like we did previously.
\begin{minted}{text} 
rosrun |\pkgname\hspace{3mm}\nodname \hspace{1mm}|/cmd_vel:=/turtle1/cmd_vel
\end{minted}
 
 \end{description}
 
\newpage

\item Now create a {\bf subscriber node} in the same package as the previous node. You can follow the tutorial \href{http://wiki.ros.org/ROS/Tutorials/WritingPublisherSubscriber(c++)} {here}. 

\begin{description}
\item [Step 1:] Use the code below called {\bf ttu\_subscriber.cpp} to start.\\

\begin{lstlisting}

#include "ros/ros.h"
#include "std_msgs/String.h"
#include "geometry_msgs/Twist.h"
/**
* This tutorial demonstrates simple receipt of messages over the ROS system.
*/
void dataCallback(const geometry_msgs::Twist::ConstPtr& msg)
{
	ROS_INFO("I heard: [%f]", msg->linear.x);
}
int main(int argc, char **argv)
{
	ros::init(argc, argv, "ttu_subscriber");
	ros::NodeHandle n;
	ros::Subscriber sub = n.subscribe("/cmd_vel", 1000, dataCallback);
	ros::spin();
	return 0;
}

\end{lstlisting}


\item [Step 2:] Modify the CMakeLists.txt file as you did previously. \\\\
%\begin{minted}{text}
%gedit |\home\wspname/src/\pkgname/CMakeLists.txt|
%\end{minted}

\item [Step 3:] Compile the new subscriber node using catkin. \\\\

%\begin{minted}{text} 
%|cd \home\wspname|
%\end{minted}

%\begin{minted}{text} 
%catkin_make
%\end{minted}

\item [Step 4:] Test the new node. Does it work? How do you know?\\

%\begin{minted}{text} 
%rosrun |\pkgname\hspace{3mm}\nodname| 
%\end{minted}

\end{description}
\vspace*{20mm}
\item Whew, that was quite alot. Please let me know if you have any questions. If you want more try this \href{http://wiki.ros.org/joy/Tutorials/WritingTeleopNode}{JoyStick Teleop Node} for use with a Linux compatable joystick.\\ 

\end{itemize}
\end{document}

